\qnhead{2. Configuring the Webapp}

In this section and the next, we use by default tomcat6. If you are to use tomcat7, replace all occurrences of tomcat6 by tomcat7.

In order to properly configure the application, we need to stop tomcat:

\bigskip
\begin{raggedleft}
\framebox[\linewidth][l]{
    \texttt{
        \$ sudo service tomcat6 stop
    }
}
\end{raggedleft}

\subsection{Edit Tomcat Properties}

The application needs to set a pre-defined folder, where it can read configuration settings and also writes log output into a file. That folder is set using the Tomcat property \texttt{sindice.home} to the value \texttt{/path/to/sindice/home}.

As a super-user, edit the file \texttt{/etc/default/tomcat6} and update the \texttt{JAVA\_OPTS} variable:

\bigskip
\begin{raggedleft}
\framebox[\linewidth][l]{
    \texttt{
        JAVA\_OPTS="-Dsindice.home=/path/to/sindice/home"
    }
}
\end{raggedleft}
\hfill

Set tomcat6 permissions on that folder, so that the deployed webapp can write in that folder:

\bigskip
\begin{raggedleft}
\framebox[\linewidth][l]{
    \texttt{
        \$ sudo chown -R tomcat6:tomcat6 /path/to/sindice/home
    }
}
\end{raggedleft}

\subsection{Create the Webapp .war File}

Edit the default XML config file of SparQLed:

\bigskip
\begin{raggedleft}
\framebox[\linewidth][l]{
    {\small \texttt{
        recommendation-servlet/src/main/resources/default-config.xml
    }}
}
\end{raggedleft}

Set\marginnote{\textcolor{blue}{{\bfseries recommender} and {\bfseries proxy} tags}} the URL to the endpoints loaded with the original RDF Graph and the DGS. The endpoint with the RDF Graph is set under the \texttt{proxy} tag, the one with the DGS graph is set under the \texttt{recommender} tag.

We assume here that both graphs are loaded into the same, \texttt{HTTP} endpoint:

\bigskip
\begin{raggedleft}
\framebox[\linewidth][l]{
    \parbox[r]{\linewidth}{\texttt{
        <backendArgs>http://path/to/sparql/endpoint</backendArgs>
    }}
}
\end{raggedleft}
\hfill

The \texttt{sparqled.war} file is created using the following command, and is located in the directory \texttt{recommendation-servlet/target/}:

\bigskip
\begin{raggedleft}
\framebox[\linewidth][l]{
    \parbox[r]{\linewidth}{\texttt{
        \$ mvn -{}-projects recommendation-servlet -{}-also-make package
    }}
}
\end{raggedleft}

\qnhead{3. Launching the SparQLed Webapp}
\label{sec:sparqled}

With the SPARQL endpoint ready and the configuration files correctly setup, we are now ready to launch the webapp!

\subsection{Clean Previously Installed Webapp}

Remove a previously deployed SparQLed webapp, in order to avoid unexpected issues:

\bigskip
\begin{raggedleft}
\framebox[\linewidth][l]{
    \parbox[r]{\linewidth}{\texttt{
        \$ rm -rf /path/to/sindice/home/sparqled\\\hspace{0.02cm}
        \$ rm -rf \$CATALINA\_BASE/webapps/sparqled*
    }}
}\marginnote[-0.2cm]{\textcolor{blue}{On Ubuntu, \$CATALINA\_BASE=/var/lib/tomcat6/}}
\end{raggedleft}

\subsection{Run Tomcat}

Copy the \texttt{sparqled.war} file to the tomcat base directory and start tomcat:

\bigskip
\begin{raggedleft}
\framebox[\linewidth][l]{
    \parbox[r]{\linewidth}{\texttt{
        \$ sudo cp recommendation-servlet/target/sparqled.war \$CATALINA\_BASE/webapps/\\\hspace{0.002cm}
        \$ sudo service tomcat6 start
    }}
}
\end{raggedleft}

